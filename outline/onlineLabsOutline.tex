\documentclass{article}

\usepackage[margin=1in]{geometry}
\usepackage[colorlinks=true]{hyperref}
\usepackage{xcolor}
\usepackage{enumitem}
\setlist{nosep}

\title{Introductory Physics Labs:  A Tale of Two Transformations}
\author{Steven F.\ Wolf, Mark W.\ Sprague}
\date{\today}

\begin{document}
\maketitle

\section{Introduction}

Here we set up the rationale behind our initial transformation, specifically, we wanted our
labs to engage students in Scientific Practices.  Then, when COVID came here, we didn't want to
lose the progress we made with this transformation, so we describe our need to develop an
online curriculum that accommodates our students while duplicating many of the face-to-face
elements that we found essential to this curriculum.

\section{Instructional Context}

\begin{itemize}
  \item Institutional profile.
  \begin{itemize}
    \item Student gender and demographics
    \item University environment: rural, many communities without high speed internet
  \end{itemize}
  \item Describe student population in labs/lecture courses served.  Our lab sequence is at the
  intro level, serving both Algebra-based and Calculus-based physics lecture courses.
  \item Compare course demographics with university demographics.
\end{itemize}


\section{Transformation \#1: Argument Driven Inquiry}

Brief lit review of Argument Driven Inquiry (ADI), including some of our prior work (PERC 2019
proceedings paper).  Describe the stages of ADI and how we implement them.

\section{Transformation \#2: Online Adaptation}

How did we adapt the ADI curriculum to online formats?
\begin{itemize}
  \item Stage 1: Finishing Spring 2020 semester online
  \item Stage 2: Fall 2020 and beyond - lab kits
\end{itemize}

\section{Conclusion}
We plan to highlight some of our successes and challenges that we have encountered, as well as
discuss our outlook for online labs in a post-pandemic world.


\end{document}
