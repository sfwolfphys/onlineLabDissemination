\documentclass[11pt, oneside]{article}   	
\usepackage[letterpaper, margin=1in]{geometry}  
\usepackage[parfill]{parskip}  
\usepackage{graphicx}
\usepackage{amssymb}
\usepackage{siunitx}
\usepackage[colorlinks=true, urlcolor=blue, linkcolor=black]{hyperref}
\thispagestyle{empty}
%\renewcommand*{\thefootnote}{\fnsymbol{footnote}}
%SetFonts

%SetFonts



\begin{document}
\begin{center}
  {\Large Investigation: Measuring and characterizing human reaction time}\\
  Joint SACS AAPT/NCS AAPT Fall meeting - Workshop
  November 7, 2020
\end{center}

\section*{Investigation Groups}
Everyone has been assigned to an investigation group of three or four people. Each group will
interact in a breakout session where they will collaborate on the investigation.

\section*{Guiding Question}
Do two people have the same reaction time?

\section*{Experiment}
Your task is to design and carry out a procedure to measure a person's reaction time using the
online reaction timer at
\href{https://nrich.maths.org/reactiontimer}{https://nrich.maths.org/reactiontimer} and compare
two measured values accounting for uncertainties in order to answer the guiding question.

\section*{Argumentation}
Each group will prepare a three-slide PowerPoint (PPT) presentation with their argument. The
presentation slides should be:
\begin{enumerate}
  \item Description of procedure.  What did you measure, and how did you measure it?
  \item Results. This can include a graph, a table, or a result from your measurements.
  \item Argument. Answer the guiding question. Use your results to justify your argument.
\end{enumerate}
Select a presenter from each group. The presenter will give the group's PowerPoint during the
argumentation session. The other members of the group will be responsible for contributing to
the discussion about the each groups' presentations.

\end{document}
