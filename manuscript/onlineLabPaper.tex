\documentclass[aip, numerical, preprint]{revtex4-2}

\usepackage{graphicx}% Include figure files
\usepackage{dcolumn}% Align table columns on decimal point
\usepackage{bm}% bold math


\usepackage[colorlinks=true]{hyperref}
\usepackage{xcolor}
\usepackage{enumitem}
\setlist{nosep}

\begin{document}
\title{Introductory Physics Labs:  A Tale of Two Transformations}

\author{Steven F.\ Wolf}
\affiliation{East Carolina University Department of Physics 1000 E.\ 5\textsuperscript{th}
  street, Greenville, NC 27858 USA}
\author{Mark W.\ Sprague}
\affiliation{East Carolina University Department of Physics 1000 E.\ 5\textsuperscript{th}
  street, Greenville, NC 27858 USA}

\date{\today}


\begin{abstract}
  A key problem facing physics departments, especially given the current global pandemic, is
  how can we engage students in our laboratory courses while maintaining appropriate social
  distancing and hygiene standards.  One solution to this problem is to move the labs to an
  online format.  But how can we engage our students in online labs with a curriculum that
  privileges science practices?  We have created an intro physics lab curriculum that engages
  students in science practices and are implementing it online.
\end{abstract}

\maketitle

\section{Introduction}
\textcolor{blue}{SFW mostly writes this.}

Here we set up the rationale behind our initial transformation.  I see this as two-fold:
\begin{enumerate}
  \item We hadn't updated our labs in decades, and they utilized outdated technology and
  pedagogies.
  \item We wanted to give students an opportunity to delve into science practices.
\end{enumerate}
We have been seeking institutional (SFW's Teaching Grant) and national (NSF Award \#) resources
to support our initial transformation.

Then we've got a pandemic which means online labs were an immediate default. However, a lack of
online labs is also a barrier for DE students' completion of a degree.  So online lab curricula
would fill an institutional need.

\section{Instructional Context}
\textcolor{blue}{SFW mostly writes this.}

We should also talk about the specific context of our labs:
\begin{itemize}
  \item Discuss institutional profile.
  \item Discuss student population in labs/lecture courses served.
  \item Compare course demographics with university demographics.
\end{itemize}

\section{Transformation \#1: Argument Driven Inquiry}
\textcolor{blue}{SFW mostly writes this.}

Brief lit review of ADI, including some of our prior work (PERC 2019 proceedings paper).
Describe the stages of ADI and how we implement them.

\section{Transformation \#2: Online Adaptation}
\textcolor{blue}{MWS mostly writes this.}

How did we adapt the ADI curriculum to online formats?
\begin{itemize}
  \item Stage 1: Finishing Spring 2020 semester online
  \item Stage 2: Fall 2020 and beyond - lab kits
\end{itemize}
Lab kit composition

\section{Conclusion}
\textcolor{blue}{SFW and MWS write this together.}

We did it, and you can too!  We should have a resources page/EPAPS for this paper.

\section{Notes, not part of outline}
Most TPT papers are \emph{short}---on the order of 2000 words.  For example the following two
papers are 3 and 4 pages respectively:
\begin{itemize}
  \item \url{https://doi.org/10.1119/1.5145407}
  \item \url{https://doi.org/10.1119/1.5145524}
\end{itemize}
So we don't need to kill ourselves on a long-form manuscript.  I think we can turn this around
in a month or two and get this out.

Also, while I could see having Joi read this and give us some feedback, I'm not sure she needs
to be an author.  But this isn't a hard and fast stance that I have, and am willing to discuss
with you.  More to the point, I'm not sure how much she needs this paper.

\end{document}
