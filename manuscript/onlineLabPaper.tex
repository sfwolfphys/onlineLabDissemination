\documentclass[aip, numerical, preprint]{revtex4-2}

\usepackage{graphicx}% Include figure files
\usepackage{dcolumn}% Align table columns on decimal point
\usepackage{bm}% bold math
\usepackage{siunitx}

\usepackage[colorlinks=true,urlcolor=blue,citecolor=blue]{hyperref}
\usepackage{xcolor}
\usepackage{enumitem}
\setlist{nosep}

\begin{document}
\title{Introductory Physics Labs: A Tale of Two Transformations}

% \author{Steven F.\ Wolf} \affiliation{East Carolina University Department of Physics 1000 E.\
% 5\textsuperscript{th} street, Greenville, NC 27858 USA} \author{Mark W.\ Sprague}
% \affiliation{East Carolina University Department of Physics 1000 E.\ 5\textsuperscript{th}
% street, Greenville, NC 27858 USA}

\author{First Author} \affiliation{Masked Institution} \author{Second Author}
\affiliation{Masked Institution}

\date{\today}


\begin{abstract}
  A key problem facing physics departments, especially given the current global pandemic, is
  how can we engage students in our laboratory courses while maintaining appropriate social
  distancing and hygiene standards.  One solution to this problem is to move the labs to an
  online format.  But how can we do this while \textit{also} engaging our students in online
  labs with a curriculum that privileges science practices?  We have created an intro physics
  lab curriculum that engages students in science practices and are implementing it online.
\end{abstract}

\maketitle

\section{Introduction}
Scientific truth is discovered through the practice of doing science.  In fact, the National
Research Council in it's report on Discipline Based Education Research (DBER) states, ``Without
[scientific] investigative practices, there would be no new scientific and engineering
knowledge.''\cite{DBERreport} However, too often, laboratory courses do not meaningfully engage
students in scientific practices, instead reproducing classical experiments.\cite{PCAST12} This
is part of a long-standing tradition of calling for authentic engagement in scientific
practices which pre-dates Lorentz transformations.\cite{AAAS1881} Indeed, the AAAS stated our
goals quite succinctly, ``[Typical courses in t]he sciences\ldots are not made the means of
cultivating the observing powers, stimulating inquiry, exercising the judgment in weighing
evidence, nor of forming independent habits of thought.''\cite{AAAS1881}

The situation at our home institution in the Fall of 2016 very much mirrored the national
narrative.  The lab curriculum had been largely unchanged for about 20 years, and followed the
traditional cookbook lab paradigm.  Moreover, the equipment commonly used by students was
outdated -- triple beam balances and hand-drawn graphs were standard.  In this same year, as
part of an interdisciplinary team interested in engaging students in scientific practices as
well as assessing those scientific practices, we secured funding, both internally and from the
NSF, to change this status quo in physics, biology, and chemistry.  All three departments
transformed their labs using a pedagogical framework entitled, Argument Driven Inquiry
(ADI),\cite{Sampson2011,Walker2011,Walker2016} which has been specifically designed to engage
students in authentic science practices. We will focus on the physics transformation in this
paper.

We were satisfied with our progress when March 2020 rolled around, and with it, the mandatory
shut-down due to the global pandemic.  With students off campus, we had to abruptly shift our
delivery to an online format, and quickly realized that some of the activities that we utilized
in the face-to-face sections had to be adapted to online use.  Going forward, we've further
adapted what we are doing so that students can still participate in a hands-on laboratory
experience.  We hope these changes will be meaningful long-term as a lack of online labs is
also a barrier for DE students' completion of a degree.  So online lab curricula would fill an
institutional need.  In the next sections, we will discuss our institutional and instructional
contexts and describe each transformation in more detail.

\section{Instructional Context}
\textcolor{red}{This section needs to have the institution masked before we send it in.}

East Carolina University is a regional master's university in a rural, economically depressed
part of our state.  The university mission is focused on regional transformation and service to
this region.  The total student population in Fall 2019 was 23k.  Our physics labs (Physics 1
lab and Physics 2 lab) are each 1 credit courses that meet for 2 hours per week. These courses
serve both of our Calculus-based and Algebra-based physics lecture courses.  Most $(\sim 75\%)$
of the students who take our lab course are in (or have taken) the Algebra-based physics
lecture course. It also fulfills the general education requirement for science lab courses at
our university.  Table \ref{tab:gender} compares the gender of students in these lab courses
compared to the university population.  We note that the gender distribution of students in the
Physics 2 lab generally matches the university population, while the Physics 1 lab skews more
heavily male than the university population.  We also compare the race of students in these
contexts in Table \textcolor{red}{not ready yet}\ldots

Students who take these labs are most often science majors, especially in the health sciences
(e.g., Biology, Exercise Science), but this lab is also taken by our majors.  In Fall 2019 we
had 409 students in the Physics 1 lab, and in Spring 2020 (at Census Day, before the pandemic
changed enrollments) we had 256 students in the Physics 2 lab.  In both labs, our enrollment is
capped at 22 students, due to the constraints of our laboratory classroom.

Labs are supervised by MWS, but each section is run by Graduate TAs (GTAs). The transformed
curriculum was jointly written by both of the authors.  TA training was greatly enhanced when
the new lab curriculum was put in place.  SFW and MWS, along with colleagues in Biology and
Chemistry, run a training for all GTAs in Biology, Chemistry, Geology, and Physics as these
disciplines are all using the same curricular format for their labs.  MWS runs a weekly prep
meeting with the TAs, and SFW and the research team have supervised various aspects of the
transformation especially important for research such as curricular implementation \cite{AL's
  paper} and assessment practices \cite{Wolf2019}.


\begin{table}
  \centering
  \begin{tabular}{l|ll|l}
    \hline \hline							
    &	\multicolumn{2}{c|}{Course}\\
    Gender	&	Physics 1	&	Physics 2	&	University	\\ \hline
    Female	&	41.8\%	&	60.9\%	&	57.1\%	\\
    Male	&	58.2\%	&	39.1\%	&	42.9\%	\\
    \hline \hline							
  \end{tabular}
  \caption{Gender breakdown of students in the Fall 2019 of the Physics 1 lab, the Spring 2020
    Physics 1 lab, and the undergraduate poplulation of the university at large in Fall 2019.
    Note: The university only collected binary gender data }
  \label{tab:gender}
\end{table}

\section{Transformation \#1: Argument Driven Inquiry}
Argument Driven Inquiry (ADI) is a pedagogical method that was developed to make science labs
in school better match what actually transpires in authentic research settings
\cite{Walker2011,Sampson2011,Walker2016}.  In traditional labs, we often ask students to answer
questions, or measure quantities, that are well-known using procedures that have been refined
over time.  For example, in physics we often have students measure $g$ using a well-known
method (such as a spark timer).  In ADI labs, we give students a question to answer (e.g.,
``Does the force a fan exerts on a cart depend on the mass of the cart?''), allow them to
design a procedure for an experiment that will answer that question, collect and analyze their
data, and make claims based on their data.  We accomplish this by having students engage in a
three-part cycle:
\begin{itemize}
  \item Pre-Lab - Introducing the context
  \item Proposal Development and Data Collection
  \item Argumentation and Peer Review
\end{itemize}
In the sections below, we will comment on each part of the cycle, focusing on both students and
GTAs.

\subsection{Week 1: Pre-Lab}
Part of the goal of any laboratory is to help students develop a familiarity with certain tools
and equipment common to the lab.  The purpose of the pre-lab reading and activity is to give
students familiarity with the equipment they will be using.  For the investigation that I
described previously, ``Does the force a fan exerts on a cart depend on the mass of the cart?''
we need students to become familiar with a few things, such as what we mean by a cart, and give
them some tools that they need to be able to answer this question.  As a part of the pre-lab
reading, students are given reading about linear regression---both how to run it in Excel, and
how to interpret the output, and how we can estimate instantaneous velocity given position
vs. time data. For this experiment, we give them a cart (leaving the fan off for now) and have
them measure the acceleration of that cart down a ramp using an ultrasonic motion detector to
collect position vs.\ time data for this motion.  Students work in pairs to do this work, and
turn in a brief summary of this work on the course management system.  For this investigation,
it is the first time that they create graphs, so we set norms for graphing such as helping them
decide which variable to put on the x-axis and the y-axis as well as including units.  Pre-lab
assignments are designed to be graded in less than 1 minute/paper based on a rubric we provide
to both students and GTAs.

\subsection{Week 2: Proposal Development and Data Collection}
During this week, students work in groups of 4 to develop a proposal, get it approved by a GTA,
and then collect and analyze their data.  We have a format for proposal development that asks
students to link the scientific concept being studied---in this example case, Newton's Second
Law---to the data that they will collect.  Students also need to propose a plan for analyzing
their data and minimizing potential sources of error (such as friction).  Once students have an
approved proposal, they begin collecting and analyzing their data.  This may lead them to
refine or revise their method (which they are encouraged to do).  Grading for the GTAs this
week is minimal (full credit for completing the proposal).  However, this is the most critical
week's for GTA training.  GTAs have to be clear that the goal is to allow a diverse number of
ways of collecting/analyzing this data.  How much does the mass of the cart need to vary?  How
many different cart masses do the students need to take?  How should students deal with
friction? All of these are questions that we want the \textit{students} to answer.  And in some
sense, we will be disappointed if all of the students come up with the same answer.
Experimental methods certainly have an impact on experimental results.

\subsection{Week 3: Argumentation and Peer Review}
During this week, the students begin by sharing their results.  Students finish analyzing their
data during the prior week, and prepare a ``poster'' on a whiteboard.  One person from the
group stays at the whiteboard while the other group members (travelers) go to other groups.
The travelers learn what other groups did, and compare to their results in a structured
argumentation setting.  Groups spend about 3-5 minutes at a poster, and rotate so that they see
at least 3 other posters.  After the travelers return, critically, the groups spend time
discussing what they saw and reflecting on their own results.  What is exciting is that
sometimes, these interactions actually drive groups to change their claims \cite{Walker2019}.

Students write a shortened lab report consisting of three sections: An introduction to the
scientific concept, a description of the experimental method or procedure, and a discussion
section laying out their results, and the evidence and justification supporting their results. 
The authors have created different peer review calibration videos for the students to watch, so
that they can look for the same things that we are going to be grading them on when we get
their final reports.  Students complete their lab report and turn it in by the end of the day.
Then, students take part in double-confidential peer review.  We facilitate this online.  They
read and reply to two different lab reports.  Finally, they get their feedback and use this to
revise their own lab report.  At the end of this week, students turn in their revised,
peer-reviewed final lab report.  This is the heaviest week of GTA grading.  We provide a rubric
for lab report grading, and each lab report generally takes about 3-5 minutes to grade.

\section{Transformation \#2: Online Adaptation}

\subsection{Sudden Transition to Online Instruction--Spring 2020}

In March 2020 the COVID 19 pandemic forced most universities, including East Carolina
University, to move all their classes online. Our General Physics I and II laboratories had
completed two out of four full investigations and the pre-lab for the third investigation
face-to-face. We were forced to find a way to engage students in an online format while
preserving the nature of the ADI laboratory experience. The face-to-face activities that we
moved online were the Investigation 3 Proposal and Argumentation and the Investigation 4
Pre-Lab, Proposal, and Argumentation. In addition, we gave our laboratory practical exam
online. We required student investigation groups to find a method for online collaboration in
which everyone in the group could participate. Some groups used Cisco WebEx (video interaction
platform licensed by our university) sessions, some used a group chat, and others used FaceTime
or other online communication applications.

During the first week of online classes each group produced their proposal and it on a Canvas
Discussion in a proposal form for approval. The teaching assistant (TA) reviewed proposals and
provided feedback or approval. Groups used the TA feedback to revise their proposals until they
were approved. Most proposals obtained approval after two or three revisions, but some required
as many as seven revisions to obtain approval. The TAs and students found communication about
proposals much more difficult online than in face-to-face classes.

Investigation 3 in the General Physics Laboratory I course was a study of the periodic motion
of a mass hanging from a spring in which the students were asked to determine when the spring's
mass must be considered as a contribution to the period. Before the transition to online
classes the students had completed a pre-lab activity in which they measured the period of a
mass on a spring. For this investigation we provided videos of various masses oscillating on
springs (100 oscillations each for 10 different masses and for the spring oscillating with no
mass) and photographs of each spring and each mass on a balance. We posted video and photograph
sets for six different springs so groups in the same lab section would each have different
springs to study.

Investigation 3 for the General Physics Laboratory II course was a study of light diffraction
in which the students were asked to determine whether hairs from two individuals had the same
diameter. Before the transition to online classes, the students had completed a pre-lab
activity in which they determined the width of a single slit by measuring the diffraction
pattern. We collected hair samples from several people and posted photographs of the
diffraction pattens of the hairs. Each photograph had a ruler at the bottom for the students to
use as a length scale. We also provided photographs of the positions of the holder and screen
on the optics bench. We gave the students a tutorial on using the ImageJ
application\citep{schrasetal12} to measure distances in a photograph.

When the TA approved a group's proposal, they assigned the group a data set for the
investigation, and the group began its measurements and analysis. We provided measurements to
the students in the most raw form possible to require them to make decisions about data
collection and analysis. Each group was required to complete its analysis and create a
three-slide presentation for the argumentation session the following week. The first slide was
a description of their measurements. Te second slide was a presentation of the results,
including a graph or table, and the third slide was their argument, based on their result.

The argumentation session was held in a Cisco WebEx session during the lab session the week
after the proposal session. One member of each group gave the presentation, which was followed
by questions. Students received credit for giving presentations, asking meaningful questions,
and responding to questions.  Following the argumentation session students submitted individual
draft reports, peer-reviewed each other's drafts, and submitted final reports in the same
manner used for the face-to-face investigations.

Investigation 4 in the General Physics Laboratory I course was about collisions, and students
were asked to determine whether a collision between two marbles was elastic. Our original plans
were to have the students video marble collisions in lab and analyze them using the Tracker
application\citep{bro2009}. When the labs went online, we provided the students with several
videos of marble collisions and asked the students to install Tracker on their computers for
analysis.  We adapted the Investigation 4 Pre-Lab assignment so students could perform them on
their own computers. The originally-planned pre-lab was an analysis of a video using
Tracker. This activity required only a few changes from the face-to-face pre-lab assignment. We
conducted the proposal session as described for Investigation 3 and assigned each group one of
six videos of colliding marbles to analyze along with mass measurements of the marbles in the
videos. The argumentation session and the remainder of the investigation were conducted in the
same manner as Investigation 3.

Investigation 4 in the General Physics Laboratory II course was a study of radioactive
decay. The pre-lab for the face-to-face course is a simulation of radioactive decay using dice
in which the students roll several dice and remove all the dice with one dot showing. We wrote
a GlowScript\citep{glowscript} program to ``roll'' randomized virtual dice so they could
perform the same activity using this simulation on their computers. For the investigation the
students were asked to determine which isotope was most common in the nuclear decay of a copper
disk that had been exposed to low-energy neutron radiation. We measured radiation counts for
several disks and also background levels and provided students with \SI{30}{s} counts vs.\ time
in CSV files for analysis. As in the other course, the investigation was conducted in the same
manner as Investigation 3.


We administered the lab practical exams for both courses in Canvas using GlowScript simulations
embedded in Canvas assignments. Students made measurements on the simulation and used their
results to make an argument answering a guiding question.

We encountered many problems with the move to online laboratories. Our students had not
selected an online class, and many were not prepared for the sudden transition from
face-to-face to online classes. Many students could not or did not attend the online WebEx
sessions or participate with their assigned groups. Some students had to get jobs when they
returned home, and others did not have access to high-speed internet. We removed
non-participating students from groups and gave them an opportunity to make up their missed
work asynchronously. Less than \SI{50}{\percent} of the students in the make-up groups
completed their work. We resorted to dropping the lowest investigation for the course.

Some students in the course did not have access to computers capable of running Tracker or
ImageJ, both of which run on Windows, Macintosh, or Linux computers but not Chromebooks or
mobile devices such as smartphones or tablets. We discovered jsTrack\citep{jstrack}, an online
Javascript web application for video analysis that runs on most computers including
Chromebooks. Students using mobile devices were not able to use jsTrack either.

\subsection{Fully Online Laboratories--Fall 2020}

\begin{table}
  \caption{\label{tab: 1251 fall} General Physics Laboratory I investigations for Fall 2020
    block schedule.}
  \begin{ruledtabular}
    \begin{tabular}{ccp{28em}}
      Investigation & Topic & Guiding Question\\
      \hline
      1 & 1-D kinematics & Does a ball rolling on an incline have the same acceleration on the way up as it does on the way down? \\
      2 & Periodic motion & At what nut position is the period of the physical pendulum equal to \SI{1.30}{s}? \\
      3 & 2-D collisions & Is the collision between two marbles elastic?
    \end{tabular}
  \end{ruledtabular}
\end{table}

We decided to hold our introductory physics laboratory courses online in the Fall 2020 semester
in order to preserve the group class interaction aspects of ADI, which would be difficult under
the social distancing requirements in place due to the pandemic.\citep{mclber20} Also our
teaching laboratories would have to operate at half capacity to maintain social distancing,
preventing us from offering the courses to the necessary number of students.  In addition to
the social distancing requirements for face-to-face class meetings, our university adopted an
eight-week block schedule, with the second block ending before the Thanksgiving holiday. Half
of the Fall 2020 courses are scheduled for the first eight-week block, and half of the courses
are scheduled for the second block. Course mapping between the originally scheduled 14-week
semester and the two eight-week blocks was based on the originally scheduled class meeting
time. In the block schedule, the one semester-hour lab courses have two two-hour meetings per
week. We determined that even though there were enough lab meetings for the synchronous
activities of four full investigations, there was not enough time between lab meetings for the
asynchronous components and timely grading for four investigations. We reduced the number of
investigations to three and added an additional pre-lab activity to each investigation. The
topics for the three investigations and the guiding questions are in Table \ref{tab: 1251 fall}
for General Physics Laboratory I and in Table \ref{tab: 1261 fall} for General Physics
Laboratory II.

\begin{table}
  \caption{\label{tab: 1261 fall} General Physics Laboratory II investigations for Fall 2020
    block schedule.}
  \begin{ruledtabular}
    \begin{tabular}{ccp{26em}}
      Investigation & Topic & Guiding Question\\
      \hline
      1 & Current and resistance & Does a light bulb behave like a resistor? \\
      2 & Time varying circuits & Do two of the lab kit capacitors have the same capacitance? \\
      3 & Diffraction of light & Are hairs from different people the same diameter?
    \end{tabular}
  \end{ruledtabular}
\end{table}

We informed the students before the course began that internet connectivity was required and
that they must have access to a computer capable of running Tracker (General Physics Laboratory
I) or ImageJ (General Physics Laboratory II). We also included these statements in the course
syllabus. Although the courses were online, most of the students were on campus allowing them
to access campus computer laboratories if they did on a computer that met the course
requirements.
  
\begin{table}
  \caption{\label{tab: 1251 lab kit} General Physics Laboratory I lab kit contents.}
  \begin{tabular}{clc}
    \hline\hline
    Quantity & Item & Investigation(s)\\
    \hline
    1 & Protractor & 1 \\
    2 & \SI{25}{mm} marble & 1, 3 \\
    1 & Tape measure with cm scale & 1, 2, 3\\
    1 & \SI{0.6}{m} threded rod & 2 \\
    1 & Eye nut & 2 \\
    3 & Nuts & 2 \\
    1 & Door hook & 2 \\
    1 & \SI{1}{m} string & 2 \\
    1 & \SI{16}{mm} marble & 3 \\
    \hline\hline
  \end{tabular}
\end{table}

We developed lab kits with supplies that allowed the students to perform the investigations
outside the teaching laboratory. We purchased the lab kit items in collaboration with our
campus bookstore, and the students purchased the lab kits from the bookstore. We ordered lab
kit items in bulk and where possible directly from manufacturers to reduce the costs of the
items. Each General Physics Laboratory I kit cost ???, and each General Physics Laboratory II
kit cost ???. Table \ref{tab: 1251 lab kit} shows the lab kit contents for General Physics
Laboratory I, and Table \ref{tab: 1261 lab kit} shows the lab kit contents for General Physics
Laboratory II.

\begin{table}
  \caption{\label{tab: 1261 lab kit} General Physics Laboratory II lab kit contents.}
  \begin{tabular}{clc}
    \hline\hline
    Quantity & Item & Investigation(s)\\
    \hline
    1 & \SI{100}{\ohm} resistor & 1 \\
    1 & \SI{330}{\ohm} resistor & 1 \\
    1 &  \SI{100}{\ohm} potentiometer & 1 \\
    1 & E10 light bulb holder & 1 \\
    1 & \SI{5}{V} E10 incandescent light bulb & 1 \\
    1 & Breadboard & 1, 2, 3 \\
    1 & Breadboard power supply & 1, 2, 3 \\
    1 & USB power supply cable & 1, 2, 3 \\
    1 & Jumper wire set & 1, 2 \\
    2 & Multimeters & 1, 2 \\
    1 & Mini screwdriver for multimeters & 1,2 \\
    5 & \SI{500}{mA} fusses for multimeters & 1,2 \\
    4 & Alligator clip leads & 1, 2 \\
    1 & \SI{1}{\mega\ohm} resistor & 2 \\
    2 & \SI{100}{\micro\farad} capacitor & 2 \\
    1 & \SI{5}{V}, \SI{650}{nm} laser module & 3 \\
    1 & Tape measure with cm scale & 3 \\            
    \hline\hline
  \end{tabular}
\end{table}


\section{Conclusion}
\textcolor{blue}{SFW and MWS write this together.}

We did it, and you can too!  We should have a resources page/EPAPS for this paper.

\begin{acknowledgments}
  Redacted for review.% This work has been supported by the NSF (Award \# 1725655).
\end{acknowledgments}

\bibliography{onlinelabs.bib}


\end{document}
