\documentclass[aip, numerical, preprint]{revtex4-2}

\usepackage{graphicx}% Include figure files
\usepackage{dcolumn}% Align table columns on decimal point
\usepackage{bm}% bold math
\usepackage{siunitx}

\usepackage[colorlinks=true,urlcolor=blue,citecolor=blue]{hyperref}
\usepackage{xcolor}
\usepackage{enumitem}
\setlist{nosep}

\begin{document}
\title{Introductory Physics Labs: A Tale of Two Transformations}

% \author{Steven F.\ Wolf} \affiliation{East Carolina University Department of Physics 1000 E.\
% 5\textsuperscript{th} street, Greenville, NC 27858 USA} \author{Mark W.\ Sprague}
% \affiliation{East Carolina University Department of Physics 1000 E.\ 5\textsuperscript{th}
% street, Greenville, NC 27858 USA}

\author{First Author} \affiliation{Masked Institution}

\author{Second Author} \affiliation{Masked Institution}

\date{\today}

% There is no abstract in a TPT paper.
% \begin{abstract}
%   A key problem facing physics departments, especially given the current global pandemic, is
%   how can we engage students in our laboratory courses while maintaining appropriate social
%   distancing and hygiene standards.  One solution to this problem is to move the labs to an
%   online format.  But how can we do this while \textit{also} engaging our students in online
%   labs with a curriculum that privileges science practices?  We have created an intro physics
%   lab curriculum that engages students in science practices and are implementing it online.
% \end{abstract}

\maketitle

\section{Introduction}
A significant challenge facing physics faculty face teaching introductory labs is engaging
students in authentic science practices.\citep{national2007Rising,PCAST12,kozminski2014aapt}
Another has been highlighted given the current global pandemic---how to engage students in our
laboratory courses while maintaining appropriate social distancing and hygiene standards. We
have chosen to answer these challenges by transforming our labs\ldots\ twice. We discuss the
rationale behind the first transformation to a practice-focused curriculum.  In March 2020 we
needed to transform our labs again, this time to accommodate online learning. This paper
discusses two chief questions: ``What are we doing to engage students in science practices?''
and ``How did we make all of this work online?''

% Scientific truth is discovered through the practice of doing science.  \citeauthor{DBERreport}
% state, ``Without [scientific] investigative practices, there would be no new scientific and
% engineering knowledge.''\citep{DBERreport} Too often, our laboratory courses do not
% meaningfully engage students in scientific practices, instead reproducing classical
% experiments.\citep{PCAST12} This is part of a long-standing tradition of calling for authentic
% engagement in scientific practices which pre-dates Lorentz transformations.\citep{AAAS1881}
% Indeed, back in \citeyear{AAAS1881}, the AAAS stated our goals quite succinctly, ``[Typical
% courses in t]he sciences\ldots are not made the means of cultivating the observing powers,
% stimulating inquiry, exercising the judgment in weighing evidence, nor of forming independent
% habits of thought.''\citep{AAAS1881}

The physics labs at our institution in Fall 2016 very much mirrored the national narrative.
The lab curriculum had been largely unchanged for about 20 years. Moreover, the equipment and
techniques used by students was outdated---triple beam balances and hand-drawn graphs were
standard.  That same year, as part of an interdisciplinary team interested in engaging students
in and assessing scientific practices, we secured funding, both internally and from the NSF, to
change this \emph{status quo} in physics, biology, and chemistry.  All three departments
transformed their labs using the Argument Driven Inquiry
(ADI)\citep{Sampson2011,Walker2011,Walker2016} pedagogical framework, which has been
specifically designed to engage students in authentic science practices. ADI is very much in
the spirit of transformations such as Modeling Physics\cite{modelingBrewe2008} and the
Investigative Science Learning Environment (ISLE) \cite{etkina2007investigative}. Oftentimes,
both Modeling Physics and ISLE are implemented in studio-style or even lecture classrooms,
whereas ADI is lab-focused. The common use of hybrid courses for Modeling and ISLE imply a
shorter time scale, for example, an ISLE cycle should take place over the course of a week,
whereas the cycles that we use in ADI take place over 4 weeks. For our institutional
transformation of our physics labs with ADI, we also utilized the AAPT Lab Recommendations to
inform our curricular choices.\citep{kozminski2014aapt}

We were satisfied with our progress when the global COVID-19 pandemic forced a shutdown in
March 2020.  With students off campus, we had to abruptly shift our delivery to an online
format and quickly realized that many of the face-to-face activities had to be adapted to
online use.  Going forward to Fall 2020 (and beyond), we have further adapted what we are doing
so that students can still participate in a hands-on laboratory experience in an online
laboratory course.


\section{Instructional Context}

% Regional State University is a public doctoral granting university in a rural, economically
% depressed part of our state.  The university mission is focused on regional transformation and
% service to this region.  The total combined student population in Fall 2019 was 28,651 largely
% comprised of undergraduates (23,081). Our physics labs (Physics \textrm{I} lab and Physics
% \textrm{II} lab) are each 1 credit courses that meet for 2 hours per week. These courses serve
% both of our Calculus-based and Algebra-based physics lecture courses.  Most $(\sim 75\%)$ of
% the students who take our lab course are in (or have taken) the corresponding Algebra-based
% physics lecture course. These labs also fulfill the general education requirement for science
% lab courses at our university.

Labs are supervised by Author 2, but sections are run by Graduate TAs (GTAs). The
transformed curriculum was jointly written by both of the authors.  GTA training was also greatly
enhanced when the new lab curriculum was put in place.  Both authors, along with colleagues in
Biology and Chemistry, run training for all GTAs in Biology, Chemistry, Geology, and Physics
as these disciplines are all using the same curricular format for their labs.  Author 2 runs a
weekly prep meeting with the GTAs, and Author 1 and the research team have supervised various
aspects of the transformation especially important for research such as curricular
implementation \citep{SmithJoyner2020} and assessment grading practices \citep{Wolf2019mask}.

\section{Transformation \#1: Argument Driven Inquiry}
Argument Driven Inquiry (ADI) is a pedagogical method developed to make science labs
in school better match what transpires in authentic research settings
\citep{Walker2011,Sampson2011,Walker2016}.  In traditional labs, we often ask students to
answer questions, or measure quantities, that are well-known using procedures that have been
refined over time.  In ADI labs, we give students a guiding question to answer experimentally
(e.g., ``Does the force a fan exerts on a cart depend on the mass of the cart?''). This allows
students to design a procedure for an experiment that will answer that question, collect and
analyze their data, and make claims based on their data.  We accomplish this by having students
engage in a three-part cycle:
\begin{itemize}
  \item Pre-Lab - Introducing the context
  \item Proposal Development and Data Collection
  \item Argumentation and Peer Review
\end{itemize}
In the sections below, we will comment on each part of the cycle, focusing on both students and
GTAs. Each lab course completes at least three cycles (usually four in the Fall and Spring),
and the semester ends with a practical exam where students engage in this process on a
shortened cycle without peer feedback.\cite{Wolf2019mask} We have provided supplemental materials
with more details about each of the labs in our curriculum.

\subsection{Session 1: Pre-Lab}
Part of the goal of any laboratory is to help students develop a familiarity with certain tools
and equipment common to the lab.  The purpose of the pre-lab reading and activity is to give
students familiarity with the equipment they will be using.  For the investigation described
previously, ``Does the force a fan exerts on a cart depend on the mass of the cart?''  students 
must become familiar with a few things, such as what we mean by a cart and also obtain some 
tools to be able to answer this question.  As a part of the pre-lab
reading, students learn about linear regression---both how to run it in Excel,
how to interpret the output, and how we can estimate instantaneous velocity given position
vs.\ time data. For this experiment, we give them a cart (leaving the fan off for now) and have
them measure the acceleration of that cart down a ramp using an ultrasonic motion detector to
collect position vs.\ time data.  Students work in pairs and submit a brief
summary of this work on the learning management system (LMS). This investigation is the first
time the students create graphs, so we set norms for graphing such as helping them decide which
variable to put on the $x$-axis and the $y$-axis as well as including units.  Pre-lab assignments
are designed to be graded in less than 1 minute/submission based on a rubric we provide to both
students and GTAs.

\subsection{Session 2: Proposal Development and Data Collection}
During this session, students work in groups of 3-4 on the guiding question.  They develop a
proposal, get it approved by a GTA, and then collect and analyze data.  Our proposal development 
format asks students to link the scientific concept being studied---in
this example case, Newton's Second Law---to the data that they will collect.  Students also
must propose a plan for analyzing their data and minimizing potential sources of error (such
as friction).  Once students have an approved proposal, they begin collecting and analyzing
their data.  This may lead them to refine or revise their method (which is encouraged).  Grading 
for the GTAs this week is minimal (students get full credit for getting their
proposal approved).  However, this is the most critical week for GTA training.  GTAs have to be
clear that the goal is to allow a diverse number of ways of collecting/analyzing this
data. This can be uncomfortable for the GTAs and students alike.  Students want ``the answer,''
and GTAs want to provide one for them.  However, experimental methods certainly have an impact
on experimental results. One of the goals of this curriculum is to get students to understand
this and apply it as they develop their experimental procedures and evaluate their results.

\subsection{Session 3: Argumentation and Peer Review}
During this session, the students share their results in a poster session.  Students finish analyzing
their data during the prior week, and prepare a ``poster'' on a whiteboard.  One person from
the group stays at the whiteboard while the other group members (travelers) go to other groups.
The travelers learn what other groups did, and compare to their results in a structured
argumentation setting.  Groups spend about 3--5 minutes at a poster, and rotate so that they
see at least 3 other posters.  After the travelers return, critically, the groups spend time
discussing what they saw and reflecting on their own results.  What is exciting is that
sometimes, these interactions actually drive groups to change their claims.\citep{Walker2019}

After the session, each student submits a draft report consisting of three sections: an 
introduction to the
scientific concept, a description of the experimental method or procedure, and a discussion
section laying out their results and the evidence and justification supporting their results by the 
end of the next day.
The authors have created different peer review calibration videos for the students to watch so
they can look for the same things that we will grade when we get their
final reports.  After watching the video, students take part in online, double-confidential peer 
reviews of two different lab reports using a simplified rubric based on the grading rubric for final drafts.
Finally, students use their feedback to revise their own lab reports.  At the end of
this week, students turn in their revised, peer-reviewed final lab report.  This is the
heaviest week of GTA grading, each lab report generally takes about 3-5 minutes to grade.

\section{Transformation \#2: Online Adaptation}

\subsection{Sudden Transition to Online Instruction--Spring 2020}

In March 2020 the COVID-19 pandemic forced most universities, including ours, to move all
classes online. Our Physics 1 and 2 laboratories had completed two out of
four full investigations and the pre-lab for the third investigation face-to-face. We were
forced to find a way to engage students in an online format while preserving the nature of the
ADI laboratory experience. In addition, we gave the laboratory practical exam online. We
required student investigation groups to find a method for online collaboration in which
everyone in the group could participate. Tools students used were the LMS (Canvas)
communication tools, WebEx (video interaction platform licensed by our university),
and other online communication applications not managed by the university (\emph{e.g.}, group
chats). We have detailed the adaptations we made to the curriculum in the supplementary
materials.  One of the principles that we used while making these adaptations is that students
should exclusively use university-supported (Microsoft Office) or open source resources.

The pre-lab phase was managed one of two ways.  As this is the most prescriptive part of the lab 
intended to introduce students to data collection and analysis techniques, in-person
activities were modified for online use by creating (or using) videos or simulated using
\emph{Glowscript}.\citep{glowscript} Students could collaborate with a partner asynchronously
for this part of the project.

Proposal development also occurred online asynchronously.  Each group produced their proposal
and posted a proposal form on a discussion board for approval. The GTA reviewed proposals and
provided feedback or approval. Groups used the GTA feedback to revise their proposals until
they were approved. As with the prior face-to-face labs, the GTAs were looking for specific items in 
the proposals such as what data the students planned to collect, how they would use the data to 
answer the guiding question, what steps they would take to minimize errors, and how the students 
would quantify their uncertainty. Most proposals obtained approval after two or three revisions, but 
some required as many as seven revisions to obtain approval. When the GTA approved a group's
proposal, they assigned the group a data set for the investigation, and the group began
measurements and analysis. Groups were encouraged to revise their methods during the the 
measurement/analysis phase if the discovered a flaw or a better way to answer the guiding question. 
Based on our prior experience running the class, we prepared some
data sets appropriate to the different proposals that previous students had produced and
provided measurements to the current students in the most raw form possible to require them to
make decisions about data collection and analysis. We provided six data sets for the third investigation 
and 13 data sets for the fourth investigation for the Physics 1 laboratory. We provided seven data sets 
for the third investigation (each group was assigned two data sets at random to compare) and 14 data 
sets for the fourth investigation for the Physics 2 laboratory.

The argumentation session was held in a scheduled WebEx session
the week after the proposal session. Before this session, each group was required to complete
its analysis and create a three-slide presentation for the argumentation session: (1) a description of 
their measurements, (2) a presentation of the
results, including a graph or table, and (3) their argument based on their
results.  One member of each group gave the presentation, which was followed by
questions. Students received credit for giving presentations, asking meaningful questions, and
responding to questions.  Following the argumentation session, students submitted individual
draft reports, peer-reviewed each other's drafts, and submitted final reports in the same
manner used for the face-to-face investigations.

We administered the lab practical exams for both courses in the LMS using GlowScript simulations
embedded in LMS assignments. Students made measurements on the simulation and used their
results to make an argument answering a guiding question.

\subsection{Fully Online Laboratories--Fall 2020}

Early on, we decided to hold our introductory physics laboratory courses online in the Fall
2020 semester to preserve the group class interaction aspects of ADI, which would be
difficult under social distancing requirements in place due to the
pandemic.\citep{mclber20} Learning from some of the challenges we faced in the Spring, we
informed the students before the course began that internet connectivity was required and that
they must have access to a computer capable of running \emph{Tracker}\citep{bro2009} and 
\emph{ImageJ}\citep{schrasetal12} (used in the course). We also included these statements in the course
syllabus. Although we began the semester with the university open for socially distant
face-to-face learning, that has since ended, and we have pivoted to online instruction only for
our undergraduate students.

We developed lab kits with supplies that allowed students to perform the investigations outside
the teaching laboratory so we could provide students with a hands-on experience. We purchased
the lab kit items in collaboration with our campus bookstore, and the students purchased the
lab kits from the bookstore. We ordered lab kit items in bulk and, where possible, directly from
manufacturers to reduce the costs of the items. Each Physics 1 kit cost \$20.60 and each
Physics 2 kit cost \$39.00. Students payed for these kits in place of a lab manual, which we
provide online for this course. Lab kit materials were used for each investigation. The two video 
analysis investigations in the Physics 1 laboratory (comparing the accelerations of a marble 
rolling up and down an incline and a marble collision in two dimensions) required the students to 
record their own videos of marbles from their lab kits using their smartphone cameras. The image 
analysis investigation in the 
Physics 2 laboratory required students to cut their own slits in aluminum foil and also create mounts 
for hairs to produce diffraction patterns with lab-kit lasers. The online adaptation of the face-to-face 
ADI curriculum has changed slightly. Pre-labs are done in groups asynchronously.  Proposals are 
developed in groups during the lab period, and scheduled argumentation sessions occur via 
WebEx during the lab period.

\section{Discussion}
We have observed a great number of successes for these transformations.  First and foremost, we
indeed have a lab curriculum that is aligned with national goals and standards. Furthermore, we
have successfully deployed this online.  Students are engaging with science practices as
determined by our practical assessments, and our department culture around laboratory course
instruction is invigorated. Psychometric assessment of our students engagement with science
practices is ongoing.

The transition to online learning has not gone without hiccups.  Technology and access are two
key factors for success in online learning environments, which were a barrier for many of our
students.  Our institution implemented a number of academic measures intended to support our
students during the pandemic, including extending the withdrawal date and allowing students to
be graded pass/fail after seeing their final grade during each of the Spring 2020 and Fall 2020
semesters. There have also been technical supports put in place at the institution such as
extended laptop loan programs and remote proctoring was paid for by a grant through the
university system. Additionally, our institution's Center for Survey Research has published
some findings in August 2020 that explain some of our observed
issues.\citep{ECUcovidSurveyRedacted} The GTAs and students found communication about proposals
much more difficult online than in face-to-face classes.  About \SI{10}{\percent} to
\SI{20}{\percent} of our students did not have reliable internet access after the transition to
online learning in the spring.\citep{ECUcovidSurveyRedacted} Also, up to \SI{8}{\percent} of
students did not have access to a computer once they left campus.\citep{ECUcovidSurveyRedacted}
Additionally, we noticed that some students in the course did not have access to computers
capable of running \emph{Tracker} or \emph{ImageJ}, both of which run on Windows, Macintosh, or
Linux computers but not Chromebooks or mobile devices. We discovered
\emph{jsTrack}\citep{jstrack}, an online Javascript web application for video analysis that
runs on most computers including Chromebooks, but not mobile devices.

Many students could not or did not attend the online WebEx sessions or participate with their
assigned groups.  Once they left campus, many students found they had increased work or school
responsibilities (\SI{42.1}{\percent}) or additional family responsibilities
(\SI{59.0}{\percent}).\citep{ECUcovidSurveyRedacted} We removed non-participating students from
groups and gave them an opportunity to make up their missed work asynchronously. Less than
\SI{50}{\percent} of the students in the make-up groups completed their work.

Finally, we hope these changes will be meaningful long-term as a lack of online labs is also a
barrier for distance education students' completion of a degree so online lab curricula would
fill an institutional need.


\begin{acknowledgments}
  Redacted for review.% This work has been supported by the NSF (Award \# 1725655). We would
                      % also like to thank the Physics GTAs who have helped us develop
                      % resources and tutorials for students during this transition.
\end{acknowledgments}

\bibliography{onlinelabs.bib}


\end{document}
