\documentclass[aip, numerical, preprint]{revtex4-2}

\usepackage{graphicx}% Include figure files
\usepackage{dcolumn}% Align table columns on decimal point
\usepackage{bm}% bold math
\usepackage{siunitx}

\usepackage[colorlinks=true]{hyperref}
\usepackage{xcolor}
\usepackage{enumitem}
\setlist{nosep}

\begin{document}
\title{Introductory Physics Labs:  A Tale of Two Transformations}

\author{Steven F.\ Wolf}
\affiliation{East Carolina University Department of Physics 1000 E.\ 5\textsuperscript{th}
  street, Greenville, NC 27858 USA}
\author{Mark W.\ Sprague}
\affiliation{East Carolina University Department of Physics 1000 E.\ 5\textsuperscript{th}
  street, Greenville, NC 27858 USA}

\date{\today}


\begin{abstract}
  A key problem facing physics departments, especially given the current global pandemic, is
  how can we engage students in our laboratory courses while maintaining appropriate social
  distancing and hygiene standards.  One solution to this problem is to move the labs to an
  online format.  But how can we engage our students in online labs with a curriculum that
  privileges science practices?  We have created an intro physics lab curriculum that engages
  students in science practices and are implementing it online.
\end{abstract}

\maketitle

\section{Introduction}
\textcolor{blue}{SFW mostly writes this.}

Here we set up the rationale behind our initial transformation.  I see this as two-fold:
\begin{enumerate}
  \item We hadn't updated our labs in decades, and they utilized outdated technology and
  pedagogies.
  \item We wanted to give students an opportunity to delve into science practices.
\end{enumerate}
We have been seeking institutional (SFW's Teaching Grant) and national (NSF Award \#) resources
to support our initial transformation.

Then we've got a pandemic which means online labs were an immediate default. However, a lack of
online labs is also a barrier for DE students' completion of a degree.  So online lab curricula
would fill an institutional need.

\section{Instructional Context}
\textcolor{blue}{SFW mostly writes this.}

We should also talk about the specific context of our labs:
\begin{itemize}
  \item Discuss institutional profile.
  \item Discuss student population in labs/lecture courses served.
  \item Compare course demographics with university demographics.
\end{itemize}

\section{Transformation \#1: Argument Driven Inquiry}
\textcolor{blue}{SFW mostly writes this.}

Brief lit review of ADI, including some of our prior work (PERC 2019 proceedings paper).
Describe the stages of ADI and how we implement them.

\section{Transformation \#2: Online Adaptation}
\subsection{Sudden Transition to Online Instruction--Spring 2020}

In March 2020 the COVID 19 pandemic forced most universities, including East Carolina University, to move all their classes online. Our General Physics I and II laboratories had completed two out of four full investigations and the pre-lab for the third investigation face-to-face. We were forced to find a way to engage students in an online format while preserving the nature of the ADI laboratory experience. The face-to-face activities that we moved online were the Investigation 3 Proposal and Argumentation and the Investigation 4 Pre-Lab, Proposal, and Argumentation. In addition, we gave our laboratory practical exam online. We required student investigation groups to find a method for online collaboration in which everyone in the group could participate. Some groups used Cisco WebEx (video interaction platform licensed by our university) sessions, some used a group chat, and others used FaceTime or other online communication applications. 

During the first week of online classes each group produced a proposal and posted a proposal form on a Canvas group discussion page for approval. The TA reviewed proposals a provided feedback or approval. Groups used the TA feedback to revise their proposals until they were approved. Most proposals obtained approval after two or three revisions, but some required as many as seven revisions to obtain approval. The TAs and students found communication about proposals much more difficult online than in face-to-face classes. 

Investigation 3 in the General Physics Laboratory I course was a study of the period of a mass hanging from a spring in which the students were asked to determine when the spring's mass must be considered as a contribution to the period. Before the transition to online classes the students had completed a pre-lab activity in which they measured the period of a mass on a spring. For this investigation we provided videos of various masses oscillating on springs (100 oscillations for 10 different masses and the spring oscillating with no mass) and a photograph of each spring on a balance. We posted data sets for six different springs so groups in the same lab section would each have different springs to study. 

Investigation 3 for the General Physics Laboratory II course was a study of diffraction in which the students were asked to determine whether hairs from two individuals had the same diameter. We collected hair samples from several people and posted photographs of the diffraction pattens of the hairs. Each photograph had a ruler at the bottom for the students to use as a length scale. We also provided photographs of the positions of the holder and screen on the optics bench. We gave the students a tutorial on using the ImageJ application\citep{schrasetal12} to measure distances in a photograph.

When the TA approved a group's  proposal, they assigned the group a data set for the investigation, and the group began its measurements and analysis. We provided measurements to the students in the most raw form possible to require them to make decisions about data collection and analysis. Each group was required to complete its analysis and create a three-slide presentation for the argumentation session the following week. The first slide was a description of their measurements. Te second slide was a presentation of the results, including a graph or table, and the third slide was their argument, based on their result.

The argumentation session was held in a Cisco WebEx session during the lab session the week after the proposal session. One member of each group gave the presentation, which  was followed by questions. Students received credit for giving presentations, asking meaningful questions, and responding to questions.  Following the argumentation session students submitted individual draft reports, peer-reviewed each other's drafts, and submitted final reports in the same manner used for the face-to-face investigations.

Investigation 4 in the General Physics Laboratory I course was about collisions, and students were asked to determine whether a collision between two marbles was elastic. Our original plans were to have the students video marble collisions in lab and analyze them using the Tracker application\citep{bro2009}. When the labs went online, we provided the students with several videos of marble collisions and asked the students to install Tracker on their computers for analysis.  We adapted the Investigation 4 Pre-Lab assignment so students could perform them on their own computers. The originally-planned pre-lab was an analysis of a video using Tracker. This activity required only a few changes from the face-to-face pre-lab assignment. We conducted the proposal session as described for Investigation 3 and assigned each group one of six videos of colliding marbles to analyze along with mass measurements of the marbles in the videos. The argumentation session and the remainder of the investigation were conducted in the same manner as Investigation 3.

Investigation 4 in the General Physics Laboratory II course was a study of radioactive decay. The pre-lab for the face-to-face course is a simulation of radioactive decay using dice in which the students roll several dice and remove all the dice with one dot showing. We wrote a GlowScript\citep{glowscript} program to ``roll'' randomized virtual dice so they could perform the same activity using this simulation on their computers. For the investigation the students were asked to determine which isotope was most common in the nuclear decay of a copper disk that had been exposed to low-energy neutron radiation. We measured radiation counts for several disks and also background levels and provided students with \SI{30}{s} counts vs.\ time in CSV files for analysis. As in the other course, the investigation was conducted in the same manner as Investigation 3.


Practical exams go in this paragraph!!!

We encountered many problems with the move to online laboratories. Our students had not selected an online class, and many were not prepared for the sudden transition from face-to-face to online classes. Many students could not or did not attend the online WebEx sessions or participate with their assigned groups. Some students had to get jobs when they returned home, and others did not have access to high-speed internet. We removed non-participating students from groups and gave them an opportunity to make up their missed work asynchronously. Less than \SI{50}{\percent} of the students in the make-up groups completed their work. We resorted to dropping the lowest investigation for the course.

Some students in the course did not have access to computers capable of running Tracker or ImageJ, both of which run on Windows, Macintosh, or Linux computers but not Chromebooks or mobile devices such as smartphones or tablets. We discovered jsTrack\citep{jstrack}, an online Javascript web application for video analysis that runs on most computers including Chromebooks. Students using mobile devices were not able to use jsTrack either.

\subsection{Fully Online Laboratories--Fall 2020}

\begin{table}
    \caption{\label{tab: 1251 fall} General Physics Laboratory I investigations for Fall 2020 block schedule.}
    \begin{ruledtabular}
        \begin{tabular}{ccp{28em}}
            Investigation & Topic & Guiding Question\\
            \hline
            1 & 1-D kinematics & Does a ball rolling on an incline have the same acceleration on the way up as it does on the way down? \\
            2 & Periodic motion & At what nut position is the period of the physical pendulum equal to \SI{1.30}{s}? \\
            3 & 2-D collisions & Is the collision between two marbles elastic?
        \end{tabular}
    \end{ruledtabular}
\end{table}

We decided to hold our introductory physics laboratory courses online in the Fall 2020 semester in order to preserve the group class interaction aspects of ADI, which would be difficult under the social distancing requirements in place due to the pandemic.\citep{mclber20}  Also our teaching laboratories would have to operate at half capacity to maintain social distancing, preventing us from offering the courses to the necessary number of students.  In addition to the social distancing requirements for face-to-face class meetings, our university adopted an eight-week block schedule, with the second block ending before the Thanksgiving holiday. Half of the Fall 2020 courses are scheduled for the first eight-week block, and half of the courses are scheduled for the second block. Course mapping between the originally scheduled 14-week semester and the two eight-week blocks was based on the originally scheduled class meeting time. In the block schedule, the one semester-hour lab courses have two two-hour meetings per week. We determined that even though there were enough lab meetings for the synchronous activities of four full investigations, there was not enough time between lab meetings for the asynchronous components and timely grading for four investigations. We reduced the number of investigations to three and added an additional pre-lab activity to each investigation. The topics for the three investigations and the guiding questions are in Table \ref{tab: 1251 fall} for General Physics Laboratory I and in Table \ref{tab: 1261 fall} for General Physics Laboratory II.

\begin{table}
    \caption{\label{tab: 1261 fall} General Physics Laboratory II investigations for Fall 2020 block schedule.}
    \begin{ruledtabular}
        \begin{tabular}{ccp{26em}}
            Investigation & Topic & Guiding Question\\
            \hline
            1 & Current and resistance & Does a light bulb behave like a resistor? \\
            2 & Time varying circuits & Do two of the lab kit capacitors have the same capacitance? \\
            3 & Diffraction of light & Are hairs from different people the same diameter?
        \end{tabular}
    \end{ruledtabular}
\end{table}

We informed the students before the course began that internet connectivity was required and that they must have access to a computer capable of running Tracker (General Physics Laboratory I) or ImageJ (General Physics Laboratory II). We also included these statements in the course syllabus. Although the courses were online, most of the students were on campus allowing them to access campus computer laboratories if they did on a computer that met the course requirements.
  
 \begin{table}
    \caption{\label{tab: 1251 lab kit} General Physics Laboratory I lab kit contents.}
        \begin{tabular}{clc}
            \hline\hline
            Quantity & Item & Investigation(s)\\
            \hline
            1 & Protractor & 1 \\
            2 & \SI{25}{mm} marble & 1, 3 \\
            1 & Tape measure with cm scale & 1, 2, 3\\
            1 & \SI{0.6}{m} threded rod & 2 \\
            1 & Eye nut & 2 \\
            3 & Nuts & 2 \\
            1 & Door hook & 2 \\
            1 & \SI{1}{m} string & 2 \\
            1 & \SI{16}{mm} marble & 3 \\
            \hline\hline
        \end{tabular}
\end{table}

We developed lab kits with supplies that allowed the students to perform the investigations outside the teaching laboratory. We purchased the lab kit items in collaboration with our campus bookstore, and the students purchased the lab kits from the bookstore. We ordered lab kit items in bulk and where possible directly from manufacturers to reduce the costs of the items. Each General Physics Laboratory I kit cost ???, and each General Physics Laboratory II kit cost ???. Table \ref{tab: 1251 lab kit} shows the lab kit contents for General Physics Laboratory I, and Table \ref{tab: 1261 lab kit} shows the lab kit contents for General Physics Laboratory II.

 \begin{table}
    \caption{\label{tab: 1261 lab kit} General Physics Laboratory II lab kit contents.}
        \begin{tabular}{clc}
            \hline\hline
            Quantity & Item & Investigation(s)\\
            \hline
            1 & \SI{100}{\ohm} resistor & 1 \\
            1 & \SI{330}{\ohm} resistor & 1 \\
            1 &  \SI{100}{\ohm} potentiometer & 1 \\
            1 & E10 light bulb holder & 1 \\
            1 & \SI{5}{V} E10 incandescent light bulb & 1 \\
            1 & Breadboard & 1, 2, 3 \\
            1 & Breadboard power supply & 1, 2, 3 \\
            1 & USB power supply cable & 1, 2, 3 \\
            1 & Jumper wire set & 1, 2 \\
            2 & Multimeters & 1, 2 \\
            1 & Mini screwdriver for multimeters & 1,2 \\
            5 & \SI{500}{mA} fusses for multimeters & 1,2 \\
            4 & Alligator clip leads & 1, 2 \\
            1 & \SI{1}{\mega\ohm} resistor & 2 \\
            2 & \SI{100}{\micro\farad} capacitor & 2 \\
            1 & \SI{5}{V}, \SI{650}{nm} laser module & 3 \\
            1 & Tape measure with cm scale & 3 \\            
            \hline\hline
        \end{tabular}
\end{table}


\section{Conclusion}
\textcolor{blue}{SFW and MWS write this together.}

We did it, and you can too!  We should have a resources page/EPAPS for this paper.

\bibliography{onlinelabs.bib}


\end{document}
