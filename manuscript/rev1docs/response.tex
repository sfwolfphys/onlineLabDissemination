\documentclass{letter}

\usepackage[top=1in,left=1.5in,right=1.5in]{geometry}

\makeatletter
\renewcommand{\closing}[1]{\par\nobreak\vspace{\parskip}%
  \stopbreaks
  \noindent
  \ifx\@empty\fromaddress\else
  \hspace*{\longindentation}\fi
  \parbox{\indentedwidth}{\raggedright
       \ignorespaces #1\\[2\medskipamount]%
       \ifx\@empty\fromsig
           \fromname
       \else \fromsig \fi\strut}%
   \par}
\makeatother

\signature{The Authors of \textit{Introductory Physics Labs: A Tale of Two Transformations}}

\usepackage{xcolor}
\newcommand{\swsays}[1]{\textcolor{red}{SW says: #1}}

\begin{document}
\begin{letter}{Dr.\ Gary White \\ Editor: \textit{The Physics Teacher} \\ American Association
    of Physics Teachers \\ College Park, MD}

  \opening{Dear Dr.\ White and referee,}

  Thank you for your thoughtful review of our article entitled \textit{Introductory Physics
    Labs: A Tale of Two Transformations}. We have considered your review and are responding to
  your feedback in the following ways.

  \begin{description}
    \item[Institutional support for students needs] Dr.\ White wanted to know more about the
    ways that our institution attempted to support students during the pandemic.  Our office of
    student affairs developed programs and outreach to help support students such as
    disseminating CARES Act funding to students demonstrating exceptional need.  We do discuss
    academic measures taken by the university such as extending the withdrawal date, exempting
    withdrawals during the Spring 2020 and Fall 2020 semesters from the overall limit on
    withdrawals, and allowing students to be graded pass/fail after seeing their final grade
    (and having that grade not count toward their GPA).
    \item[Comparing ADI to Modeling Instruction] We've added a few sentences to the
    introduction comparing ADI to Modeling Instruction.  We also discuss ISLE (Investigative
    Science Learning Environment) as it is also a similar pedagogical approach driven by
    learning cycles.  Both Modeling and ISLE tend to be implemented in lecture courses, while
    ADI was designed to mimic the ways that scientists learn in the laboratory, and has been
    designed to be implemented in laboratory courses regardless of the approach used in
    lecture. We find it to be particularly useful in our context as students in each of our lab
    courses take two different intro physics lecture courses (algebra-based and calculus-based).
    \item[Clarity about proposals and proposal approvals] We've added a few sentences talking
    about how we generated datasets during Spring 2020 and approve proposals in general. We
    want to point out that we do allow some proposals to go forward even if they have a flaw
    once data is collected.  Students are free to revise their proposals and gather more
    data. Dealing with unexpected results can be a powerful teacher. As far as creating
    datasets goes, we tried to make these data as raw as possible, for example sharing videos
    of a object oscillating on a spring and the mass of that object rather than giving them
    mass and period data directly. \swsays{Mark, do you want to say something about how you
      chose what data to generate?}
    \item[Details about Fall 2020] For the Fall 2020 semester, all data were collected by the
    students.  This includes images and videos as appropriate, often collected with students'
    smartphones. \swsays{Mark, please fact-check the above statement.}
    \item[Details about the survey] The survey was conducted during the spring/early summer of
    2020.  Results were released to the faculty before the fall 2020 semester began.  This was
    intended to help the faculty better understand the difficulties that were impacting our
    students. The date of the survey has been added to the text in the conclusion.
  \end{description}

  Thank you for your time.
  
  \closing{Sincerely,}
    
\end{letter}


\end{document}
