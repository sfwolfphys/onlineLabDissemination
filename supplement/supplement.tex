\documentclass[aip, numerical, preprint]{revtex4-2}

\usepackage{graphicx}% Include figure files
\usepackage{dcolumn}% Align table columns on decimal point
\usepackage{bm}% bold math
\usepackage{siunitx}

\usepackage[colorlinks=true,urlcolor=blue,citecolor=blue]{hyperref}
\usepackage{xcolor}
\usepackage{enumitem}
\setlist{nosep}

\begin{document}
\title{Supplementary Materials: Introductory Physics Labs: A Tale of Two Transformations}

% \author{Steven F.\ Wolf} \affiliation{East Carolina University Department of Physics 1000 E.\
% 5\textsuperscript{th} street, Greenville, NC 27858 USA} \author{Mark W.\ Sprague}
% \affiliation{East Carolina University Department of Physics 1000 E.\ 5\textsuperscript{th}
% street, Greenville, NC 27858 USA}

\author{First Author} \affiliation{Masked Institution}

\author{Second Author} \affiliation{Masked Institution}

\date{\today}


\maketitle

This supplement is meant to accompany the paper, \textit{Introductory Physics Labs: A Tale of
  Two Transformations} being submitted to \textit{The Physics Teacher}.  It is divided into the
following sections:
\begin{description}
  \item[Additional Instructional Context] discusses some particulars of our department and
  institution that can help the reader put the scope and population of the intervention into
  context.
  \item[Transformation \#1: Argument Driven Inquiry] provides a brief summary of each of the
  experiments done in the face-to-face version of this lab course.
  \item[Transformation \#2: Online delivery] provides a summary of how we adapted or created
  new activities for students to do their labs online, and discusses some of the ways that we
  supported authentic experimentation in this online environment.
\end{description}

\section{Additional instructional context}

Regional State University is a public doctoral granting university in a rural, economically
depressed part of our state.  The university mission is focused on regional transformation and
service to this region.  The total combined student population in Fall 2019 was 28,651 largely
comprised of undergraduates (23,081).  Our physics labs (Physics 1 lab and Physics 2 lab) are
each 1 credit courses that meet for 2 hours per week. These courses serve both of our
Calculus-based and Algebra-based physics lecture courses.  Most $(\sim 75\%)$ of the students
who take our lab course are in (or have taken) the Algebra-based physics lecture course. It
also fulfills the general education requirement for science lab courses at our university.
Table \ref{tab:gender} compares the gender of students in these lab courses compared to the
university population.  We note that the gender distribution of students in the Physics 2 lab
generally matches the university population, while the Physics 1 lab skews more heavily male
than the university population.  We also compare the race of students in these contexts in
Table \ref{tab:race}.  The racial and ethnic profile of students in our lab courses is similar
to the university.

Students who take these labs are most often science majors, especially in the health sciences
(e.g., Biology, Exercise Science), but this lab is also taken by our physics majors.  In Fall
2019 we had 409 students in the Physics 1 lab, and in Spring 2020 (at Census Day, before the
pandemic changed enrollments) we had 256 students in the Physics 2 lab.  In both labs, our
enrollment is capped at 22 students, due to the constraints of our laboratory classroom.

Labs are supervised by Author 2, but each section is run by Graduate TAs (GTAs). The
transformed curriculum was jointly written by both of the authors.  GTA training was greatly
enhanced when the new lab curriculum was put in place.  Both authors, along with colleagues in
Biology and Chemistry, run a training for all GTAs in Biology, Chemistry, Geology, and Physics
as these disciplines are all using the same curricular format for their labs.  Author 2 runs a
weekly prep meeting with the GTAs, and Author 1 and the research team have supervised various
aspects of the transformation especially important for research such as curricular
implementation \citep{SmithJoyner2020} and assessment grading practices \citep{Wolf2019mask}.


\begin{table}
  \centering
  \begin{tabular}{lrrr}
    \hline \hline
                &\multicolumn{2}{c}{Course}\\
    Gender	&Physics 1 &Physics 2 &University\\
    \hline
    Female	&41.8\%	   &60.9\%    &57.1\%\\
    Male	&58.2\%	   &39.1\%    &42.9\%\\
    \hline \hline
  \end{tabular}
  \caption{Gender breakdown of students in the Fall 2019 of the Physics 1 lab, the Spring 2020
    Physics 1 lab, and the undergraduate poplulation of the university at large in Fall 2019.
    Note: The university only collected binary gender data.}
  \label{tab:gender}
\end{table}


\begin{table}
  \centering
  \begin{tabular}{lrrr}
    \hline\hline
                                            &\multicolumn{2}{c}{Course}\\
    Race/Ethnicity                          &Physics 1  &Physics 2 &University\\
    \hline
    White                                   &60.8\%     &60.8\%    &65.3\%\\
    Black or African American               &17.1\%     &15.5\%    &16.3\%\\
    Hispanic                                &7.5\%      &8.7\%     &7.5\%\\
    Two or More Races                       &7.3\%      &5.3\%     &3.8\%\\
    Unknown                                 &2.9\%      &3.0\%     &3.4\%\\
    Asian                                   &4.0\%      &6.4\%     &2.4\%\\
    American Indian or Alaska Native        &.	        &.         &0.6\%\\
    Non-Resident Alien	                    &0.4\%      &0.4\%     &0.6\%\\
    Native Hawaiian/Other Pacific Islander  &.	        &.         &0.1\%\\
    \hline\hline
  \end{tabular}
  \caption{Race/Ethnicity breakdown of students inthe Fall 2019 semester of the Physics 1 lab,
    the Spring 2020 semester of the Physics 2 lab, and the undergraduate population of the
    university at large in Fall 2019.}
  \label{tab:race}
\end{table}

\section{Transformation \#1: Argument Driven Inquiry}
In each of the sections below, we provide a brief summary of the ADI curriculum that we have
implemented in our physics labs.  We will discuss the pre-lab reading, which students are
expected to complete before they arrive in lab.  We also discuss the supplies that we use,
especially when they are novel.  Every experiment requires a computer for data analysis.
However, most supplies that we use are found in a typical college physics lab, although this
isn't universally true.  (We do a radioactive decay lab that utilizes a PuBe neutron source.)
We will also discuss some common approaches used by students in the activity.

One last thing of note: Oftentimes, we have made choices that give students a significant
hurdle to overcome (e.g., you can't neglect friction).  Student data should be messy.  There
should be some things that are unclear.  When we teach physics labs, we can give our students
the illusion that lab science is done by following well-known procedures and produce pristine
data. However, the opposite can often be true. As you read some of these descriptions, you
probably will think of 10 ways that the experiments could be done, ``better''---be it with
different techniques or equipment.  That is the point.  We want to give students an idea to
start with, and then give them the freedom to make changes, try things out, and figure out
their own way of knowing.  

\subsection{Physics \textrm{I} Investigation 1}
\begin{description}
  \item[Pre-lab reading] We provide a brief introduction to statistics, defining the terms
  average, standard deviation, and standard error.  We also discuss how to calculate those
  quantities in Excel.  We have a traditional discussion of how to use a ruler to measure
  distances, and end with a brief introduction to free-fall.
  \item[Pre-lab activity] We divide the room in half and have them measure the heights of all
  students, comparing the average heights of the students on the ``right'' and on the ``left''
  of the room.  We have them use a simple rule to compare their measurements, if the value
  $\pm$ uncertainty (standard error) overlaps for these measurements, then the measurements
  agree, otherwise, they disagree.
  \item[Supplies of note] Meter sticks and 2-meter sticks.
  \item[Argumentation Question] Do two people have the same reaction time?
  \item[Common student approaches] Most students drop meter sticks and catch them.  They
  measure the distance that the stick falls, and use the free-fall equation to relate distance
  fallen to time in free-fall.  Some find one of the myriad apps out there that will measure
  reaction time, such as the one at \url{https://humanbenchmark.com/}.  Note, not all apps are
  created equal.  
\end{description}

\subsection{Physics \textrm{I} Investigation 2}
\begin{description}
  \item[Pre-lab reading] We talk about how ultrasonic motion detectors work (schematically, not
  technically, and discuss how to calculate velocity from discrete position vs. time data. We
  finally discuss linear regression, how to interpret it on a basic level, and how it can be
  calculated in Excel.\footnote{Because we want our fit values to have uncertainties, we have
    our students install the \textit{Data Analysis Toolpack} if they are using Excel.}
  \item[Pre-lab activity] We give the students a cart and a track, have them incline that
  track, and have them measure the net force on the cart as it moves on the track.
  \item[Supplies of note] Ultrasonic motion detector, various masses, carts (with fan
  attachments), and a track.
  \item[Argumentation Question] Does the force that the fan exerts depend on the mass of the
  cart?
  \item[Common student approaches] Students measure the net force of a cart moving on a flat
  track under the influence of a fan several times, changing the mass of the cart each time.
  One particular struggle that students have is that they do not take friction into account,
  which is not negligible on this setup.
\end{description}

\subsection{Physics \textrm{I} Investigation 3}
\begin{description}
  \item[Pre-lab reading] We discuss strategies for linearizing non-linear data for well-known
  cases.
  \item[Pre-lab activity] We have the students measure the period of oscillation for a mass
  spring system, linearizing the data two ways, and comparing the spring constant for each of
  these analyses.
  \item[Supplies of note] Masses, springs, hangars, and a coiled snake (a spring with a
  non-negligible mass).
  \item[Argumentation Question] When does the spring's mass matter?
  \item[Common student approaches] Students use the traditional measure the time of 10
  oscillations and divide approach.  Other than that, experimental approaches are fairly
  uniform.  Different approaches to student analysis techniques are common and encouraged.
\end{description}

\subsection{Physics \textrm{I} Investigation 4}
\begin{description}
  \item[Pre-lab reading] We discuss collisions, and define momentum and kinetic energy.
  \item[Pre-lab activity] We have students analyze the motion of two objects on a video.  We
  have students use \textit{Tracker}\citep{bro2009} to quantify that motion.  Students then
  compare the motion of those two objects.
  \item[Supplies of note] Marbles and a meter stick.  And the students phones.  (We thought
  about getting a video camera for students to use, but realized that students already had one,
  and that camera already has a number of features built into it---such as picture
  steadying---that getting our own wasn't worth the effort.)
  \item[Argumentation Question] Is the collision between two marbles elastic?
  \item[Common student approaches] Students simply let marbles collide and videotape the
  action. Many of the problems here are on the data analysis end.  Friction is a killer here,
  and is an issue that students deal with differently in their data analyses.
\end{description}

\subsection{Physics \textrm{II} Investigation 1}
\begin{description}
  \item[Pre-lab reading] We review current, voltage, and resistance.  We also discuss how to
  use a multimeter.
  \item[Pre-lab activity] Measure the I-V curve of a resistor.
  \item[Supplies of note] Multimeters (2), lightbulb, resistor, power supply.
  \item[Argumentation Question] Does a lightbulb behave like a resistor?
  \item[Common student approaches] Students measure an I-V curve.  Sometimes the data is
  sort-of linear $(R^2\sim 0.9)$, and students have to deal with that.
\end{description}

\subsection{Physics \textrm{II} Investigation 2}
\begin{description}
  \item[Pre-lab reading] Students read about capacitance and voltage and are reminded
  \item[Pre-lab activity] Students measure the time constant for a RC circuit.
  \item[Supplies of note] We have created several RC circuit kits that are integrated so that
  the time constant is similar for all of the circuits without the bulb, then students can
  throw a switch to add a bulb to the circuit in series.  When added into the circuit, some of
  the bulbs light up, others do not. Multimeter, timer, power supply.
  \item[Argumentation Question] How does adding a light bulb in series change the behavior of a
  RC circuit?
  \item[Common student approaches] Measure the time constant for their kit.  See if $\tau$
  changes from the prelab.
\end{description}

\subsection{Physics \textrm{II} Investigation 3}
\begin{description}
  \item[Pre-lab reading] Students get a reading about diffraction and Babinet's principle.
  \item[Pre-lab activity] Students use one laser of known wavelength to measure a slit's width,
  then they use the same slit to measure the wavelength of an unknown laser.
  \item[Supplies of note] Optical bench, slit, lasers, screen.
  \item[Argumentation Question] Are hairs from different people the same diameter?
  \item[Common student approaches] Measure diffraction pattern of two hairs, use that to find
  the thickness of each and compare.
\end{description}

\subsection{Physics \textrm{II} Investigation 4}
\begin{description}
  \item[Pre-lab reading] Discussion of nuclear decay and Geiger counters.
  \item[Pre-lab activity] Students measure the ``decay constant'' of dice.
  \item[Supplies of note] We use a PuBe source to irradiate some copper pennies, and a geiger
  counter to measure the decay.
  \item[Argumentation Question] What isotope is most common in the nuclear decay of a penny?
  \item[Common student approaches] Students set up an appropriate enclosure and collect nuclear
  activity data.  The issue here is that copper has a double decay (two common isotopes), and
  the answer isn't straightforward.
\end{description}


\section{Transformation \#2: Online Adaptation}

\subsection{Sudden Transition to Online Instruction--Spring 2020}

In March 2020 the COVID-19 pandemic forced most universities, including ours, to move all their
classes online. Our General Physics I and II laboratories had completed two out of four full
investigations and the pre-lab for the third investigation face-to-face. We were forced to find
a way to engage students in an online format while preserving the nature of the ADI laboratory
experience. The face-to-face activities that we moved online were the Investigation 3 Proposal
and Argumentation and the Investigation 4 Pre-Lab, Proposal, and Argumentation. In addition, we
gave our laboratory practical exam online. We required student investigation groups to find a
method for online collaboration in which everyone in the group could participate. Some groups
used WebEx (video interaction platform licensed by our university) sessions, some used a
group chat, and others used FaceTime or other online communication applications.

During the first week of online classes each group produced their proposal and posted a
proposal form on a Canvas Discussion for approval. The GTA reviewed proposals and provided
feedback or approval. Groups used the GTA feedback to revise their proposals until they were
approved. Most proposals obtained approval after two or three revisions, but some required as
many as seven revisions to obtain approval. The GTAs and students found communication about
proposals much more difficult online than in face-to-face classes. When the GTA approved a
group's proposal, they assigned the group a data set for the investigation, and the group began
its measurements and analysis. We provided measurements to the students in the most raw form
possible to require them to make decisions about data collection and analysis.

Investigation 3 in the General Physics Laboratory I course was a study of the periodic motion
of a mass hanging from a spring in which the students were asked to determine when the spring's
mass must be considered as a contribution to the period. Before the transition to online
classes the students had completed a pre-lab activity in which they measured the period of a
mass on a spring. For this investigation we provided videos of various masses oscillating on
springs (100 oscillations each for 10 different masses and for the spring oscillating with no
mass) and photographs of each spring and each mass on a balance. We posted video and photograph
sets for six different springs so groups in the same lab section would each have different
springs to study.

Investigation 3 for the General Physics Laboratory II course was a study of light diffraction
in which the students were asked to determine whether hairs from two individuals had the same
diameter. Before the transition to online classes, the students had completed a pre-lab
activity in which they determined the width of a single slit by measuring the diffraction
pattern. We collected hair samples from several people and posted photographs of the
diffraction pattens of the hairs. Each photograph had a ruler at the bottom for the students to
use as a length scale. We also provided photographs of the positions of the holder and screen
on the optics bench. We gave the students a tutorial on using the \emph{ImageJ}
application\citep{schrasetal12} to measure distances in a photograph.

Each group was required to complete its analysis and create a three-slide presentation for the
argumentation session the following week. The first slide was a description of their
measurements. The second slide was a presentation of the results, including a graph or table,
and the third slide was their argument, based on their result.

The argumentation session was held in a WebEx session during the lab session the week
after the proposal session. One member of each group gave the presentation, which was followed
by questions. Students received credit for giving presentations, asking meaningful questions,
and responding to questions.  Following the argumentation session students submitted individual
draft reports, peer-reviewed each other's drafts, and submitted final reports in the same
manner used for the face-to-face investigations.

Investigation 4 in the General Physics Laboratory I course was about collisions, and students
were asked to determine whether a collision between two marbles was elastic. Our original plans
were to have the students video marble collisions in lab and analyze them using the
\emph{Tracker} application\citep{bro2009}, which is installed on the computers in the
laboratory. When the labs went online, we provided the students with several videos of marble
collisions and asked the students to install \emph{Tracker} on their computers for analysis.
We adapted the Investigation 4 Pre-Lab assignment so students could perform them on their own
computers. The originally-planned pre-lab was an analysis of a video using \emph{Tracker}. This
activity required only a few changes from the face-to-face pre-lab assignment. We conducted the
proposal session as described for Investigation 3 and assigned each group one of six videos of
colliding marbles to analyze along with mass measurements of the marbles in the videos. The
argumentation session and the remainder of the investigation were conducted in the same manner
as Investigation 3.

Investigation 4 in the General Physics Laboratory II course was a study of radioactive
decay. The pre-lab for the face-to-face course is a simulation of radioactive decay using dice
in which the students roll several dice and remove all the dice with one dot showing. We wrote
a GlowScript\citep{glowscript} program to ``roll'' randomized virtual dice so they could
perform the same activity using this simulation on their computers. For the investigation the
students were asked to determine which isotope was most common in the nuclear decay of a copper
disk that had been exposed to low-energy neutron radiation. We measured radiation counts for
several disks and also background levels and provided students with \SI{30}{s} counts vs.\ time
in CSV files for analysis. As in the other course, the investigation was conducted in the same
manner as Investigation 3.


We administered the lab practical exams for both courses in Canvas using GlowScript simulations
embedded in Canvas assignments. Students made measurements on the simulation and used their
results to make an argument answering a guiding question.

We encountered many problems with the move to online laboratories. Our students had not
registered for an online class, and many were not prepared for the sudden transition from
face-to-face to online classes. Many students could not or did not attend the online WebEx
sessions or participate with their assigned groups. Some students had to get jobs when they
returned home, and many students did not have access to high-speed internet. We removed
non-participating students from groups and gave them an opportunity to make up their missed
work asynchronously. Less than \SI{50}{\percent} of the students in the make-up groups
completed their work. We resorted to dropping the lowest investigation for the course.

Some students in the course did not have access to computers capable of running \emph{Tracker}
or \emph{ImageJ}, both of which run on Windows, Macintosh, or Linux computers but not
Chromebooks or mobile devices such as smartphones or tablets. We discovered
\emph{jsTrack}\citep{jstrack}, an online Javascript web application for video analysis that
runs on most computers including Chromebooks. Students using mobile devices were not able to
use \emph{jsTrack} either.

\subsection{Fully Online Laboratories--Fall 2020}

\begin{table}
  \caption{\label{tab: 1251 fall} General Physics Laboratory I investigations for Fall 2020
    block schedule.}
  \begin{tabular}{ccp{28em}}
    \hline\hline
    Investigation & Topic & Guiding Question\\
    \hline
    1 & 1-D kinematics & Does a ball rolling on an incline have the same acceleration on the way up as it does on the way down? \\
    2 & 2-D collisions & Is the collision between two marbles elastic? \\
    3 & Periodic motion & What is the nut position for which the physical pendulum small-angle period is minimum, and what is the power-law regression equation for the period of the system? \\
    \hline\hline
  \end{tabular}
\end{table}

We decided to hold our introductory physics laboratory courses online in the Fall 2020 semester
in order to preserve the group class interaction aspects of ADI, which would be difficult under
the social distancing requirements in place due to the pandemic.\citep{mclber20} Also our
teaching laboratories would have to operate at half-capacity face-to-face to maintain social
distancing, preventing us from offering the courses to the necessary number of students.  In
addition to the social distancing requirements for face-to-face class meetings, our university
adopted an eight-week block schedule, with the second block ending before the Thanksgiving
holiday. Half of the Fall 2020 courses were scheduled for the first eight-week block, and half
of the courses were scheduled for the second block. Course mapping between the originally
scheduled 14-week semester and the two eight-week blocks was based on the originally scheduled
class meeting time. In the block schedule, the one semester-hour lab courses have two two-hour
meetings per week. We determined that, even though there were enough lab meetings for the
synchronous activities of four full investigations, there was not enough time between lab
meetings for the asynchronous components and timely grading for four investigations. We reduced
the number of investigations to three and added an additional pre-lab activity to each
investigation. The topics for the three investigations and the guiding questions are in Table
\ref{tab: 1251 fall} for General Physics Laboratory I and in Table \ref{tab: 1261 fall} for
General Physics Laboratory II.

\begin{table}
  \caption{\label{tab: 1261 fall} General Physics Laboratory II investigations for Fall 2020
    block schedule.}
  \begin{tabular}{ccp{26em}}
    \hline\hline
    Investigation & Topic & Guiding Question\\
    \hline
    1 & Current and resistance & Does a light bulb behave like a resistor? \\
    2 & Time varying circuits & Do two of the lab kit capacitors have the same capacitance? \\
    3 & Diffraction of light & Are hairs from different people the same diameter? \\
    \hline\hline
  \end{tabular}
\end{table}

We informed the students before the course began that internet connectivity was required and
that they must have access to a computer capable of running \emph{Tracker} (General Physics
Laboratory I) or \emph{ImageJ} (General Physics Laboratory II). We also included these
statements in the course syllabus. Although the courses were online, most of the students were
on campus allowing them to access campus computer laboratories if they did on a computer that
met the course requirements.
  
\begin{table}
  \caption{\label{tab: 1251 lab kit} General Physics Laboratory I lab kit contents.}
  \begin{tabular}{clc}
    \hline\hline
    Quantity & Item & Investigation(s)\\
    \hline
    1 & Protractor & 1 \\
    2 & \SI{25}{mm} marble & 1, 2 \\
    1 & Tape measure with cm scale & 1, 2, 3\\
    1 & \SI{0.6}{m} threded rod & 3 \\
    1 & Eye nut & 3 \\
    3 & Nuts & 3 \\
    1 & Door hook & 3 \\
    1 & \SI{1}{m} string & 3 \\
    1 & \SI{16}{mm} marble & 2 \\
    \hline\hline
  \end{tabular}
\end{table}

We developed lab kits with supplies that allowed the students to perform the investigations
outside the teaching laboratory. We purchased the lab kit items in collaboration with our
campus bookstore, and the students purchased the lab kits from the bookstore. We ordered lab
kit items in bulk and where possible directly from manufacturers to reduce the costs of the
items. Each General Physics Laboratory I kit costs \$20.60, and each
General Physics Laboratory II kit costs \$39.00. Table \ref{tab: 1251 lab kit} shows the lab
kit contents for General Physics Laboratory I, and Table \ref{tab: 1261 lab kit} shows the lab
kit contents for General Physics Laboratory II.

\begin{table}
  \caption{\label{tab: 1261 lab kit} General Physics Laboratory II lab kit contents.}
  \begin{tabular}{clc}
    \hline\hline
    Quantity & Item & Investigation(s)\\
    \hline
    1 & \SI{100}{\ohm} resistor & 1 \\
    1 & \SI{330}{\ohm} resistor & 1 \\
    1 & \SI{100}{\ohm} potentiometer & 1 \\
    1 & E10 light bulb holder & 1 \\
    1 & \SI{5}{V} E10 incandescent light bulb & 1 \\
    1 & Breadboard & 1, 2, 3 \\
    1 & Breadboard power supply & 1, 2, 3 \\
    1 & USB power supply cable & 1, 2, 3 \\
    1 & Jumper wire set & 1, 2 \\
    2 & Multimeters & 1, 2 \\
    1 & Mini screwdriver for multimeters & 1,2 \\
    5 & \SI{500}{mA} fuses for multimeters & 1,2 \\
    4 & Alligator clip leads & 1, 2 \\
    1 & \SI{1}{\mega\ohm} resistor & 2 \\
    2 & \SI{100}{\micro\farad} capacitor & 2 \\
    1 & \SI{5}{V}, \SI{650}{nm} laser module & 3 \\
    1 & Tape measure with cm scale & 3 \\            
    \hline\hline
  \end{tabular}
\end{table}

\clearpage

\bibliography{onlinelabs.bib}


\end{document}
