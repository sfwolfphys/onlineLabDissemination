\documentclass[11pt, oneside]{article}   	
\usepackage[letterpaper, margin=1in]{geometry}  
\usepackage[parfill]{parskip}  
\usepackage{graphicx}
\usepackage{amssymb}
\usepackage{siunitx}

%SetFonts

%SetFonts

\begin{document}
\begin{center}
{\Large Plan for PHYS 1251 and PHYS 1261 Online Labs}\\
Spring 2020
\end{center}

\section{Introduction}

Due to the Coronavirus pandemic, ECU has added an additional week to spring break and moved classes online indefinitely.  The introductory physics laboratories are no exception to this.  This document contains plans to move the remainder of the Spring 2020 labs online and to make up for a lost week of instruction.

The PHYS 1251 and 1261 laboratories have completed the Investigation 3 (I-3) Pre-Lab.  This means there are six remaining in-class activities in the semester: the I-3 Proposal, I-3 Argumentation, I-4 Pre-Lab, I-4 Proposal, I-4 Argumentation, and the Practical Exam.  After the extended spring break, there are five weeks remaining in the semester.  This means that we must either compress the schedule by combining activities or reduce the number of activities.

Online activities can be synchronous, where the students interact in real time, or asynchronous, where the students must complete tasks or assignments by a deadline on their own.  This plan consists of a combination of synchronous and asynchronous activities to allow the students to interact in the ADI format online.

\section{Investigations}

Under normal circumstances, each investigation consists of three face-to-face sessions---Pre-Lab, Proposal, and Argumentation---followed by an online peer review. Following the peer review, students submit their final reports for grading.

\subsection{Pre-Lab}

There is one remaining Pre-Lab in the semester.  The I-4 Pre-Labs in both PHYS 1251 and PHYS 1261 could be completed online.  

In PHYS 1251, the Pre-Lab is a video analysis problem that uses the Tracker program to determine the velocity of a cart.  Since the Tracker program is available on most computer platforms, students could complete this outside the laboratory and submit their Pre-Lab reports online.  The Pre-Lab groups should coordinate this activity, but it could be completed asynchronously.

In PHYS 1261, the Pre-Lab is a simulation of radioactive decay using dice.  A Trinket simulation of the dice rolling activity has been developed and added to the course Canvas.  Students can run the simulation on any standards-compliant browser, including browsers on mobile devices.  The Pre-Lab groups should coordinate this activity, but it could be completed asynchronously.

\subsection{Proposal/Data Collection}

During the proposal session the lab groups must develop a proposal that must be approved by the TA before they can begin collecting data.  It is important for the students to interact with each other to form the proposal and then receive feedback form the TA about what the proposal lacks.  In the end, each group should have an approved proposal that serves as a roadmap for the rest of the investigation.

Most of the proposal development can occur asynchronously with timely TA interaction.  These are the suggested steps.

\begin{enumerate}
\item Each investigation group participates in a group discussion on Canvas to develop the proposal.  The group must form a draft proposal by a certain deadline (1 hour after the start of the lab period).  

\item At that proposal draft deadline time the TA will read the proposals from each group and offer feedback in the discussion. The TA will give the groups without approved proposals 15 minutes to revise their proposals.  

\item The TA will revisit the group discussions to either approve or suggest revisions to the proposals (using guiding questions).  Each group must have an approved proposal before they can continue to the data collection phase.

\item Each group will be given a set of data to analyze in order to answer the guiding question.  The TAs and instructors will collect several data sets in advance, and each group in a lab section will be assigned a different data set.  Where possible, the data will be in raw form, such as video of a mass oscillating on a spring, to allow the students to make measurements as well as decisions about what is important.

\item Students will be expected to analyze their data collaboratively and asynchronously.
\end{enumerate}

\subsection{Argumentation}

The argumentation session is the portion of the investigation that will take place synchronously.

\begin{enumerate}
\item Each group will prepare a three-slide PowerPoint (PPT) presentation.
    \begin{description} 
    \item[Slide 1:] Summary of the procedure used to collect and analyze data.  
    \item[Slide 2:] Data/analysis including graphs, calculations, and tables used to justify the argument.  
    \item[Slide 3:] Discussion/conclusion that answers the guiding question using the data as justification.
    \end{description}

\item The argumentation session will occur on WebEx. The students will access in through Canvas.  The TA will begin the argumentation session at 30 minutes after the lab start time.

\item Each group will provide one presenter, and the other students will be travellers.  Each presenter will present the group's PPT ``poster'' and take questions from the travellers. This will continue until every group has presented results.

\item The WebEx session will be recorded.  Students with excused absences will be given access to the recordeind and asked to write a paragraph about what they learned from it.

\item After the argumentation session, each group will be required to discuss what they learned on the group discussion board.  Groups may take more data if desired.
\end{enumerate}

\subsection{Peer Review and Report Submission}

The peer review and report submission phase will not be changed.

\section{Schedule}

Since we are losing one week of lab, the I-4 Pre-Lab will be an asynchronous homework assignment due at the beginning of the I-4 Proposal session.

\section{Practical Exam}

It is important that the practical exam include investigation design by the students.  This must be preserved if the students do their exams online.  One option is to video a large set of data and have the students ask the TA for the specific data they need.  The TA can then give them access to the appropriate videos or data sets.

\section{To Do}
This is a list of logistical items that we must handle before we are ready.

\begin{enumerate}
\item Communicate plan and expectations to students.
\item Devise a method to ensure student participation.
\item Have students install appropriate software to participate in WebEx.
\item Collect data sets and devise a scheme to distribute them to students.
\item Plan for students who are unable to connect or participate for various reasons (connectivity, illness, \emph{etc}.).
\item ...
\end{enumerate}

\end{document}  
